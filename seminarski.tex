\documentclass[12pt,a4paper]{article}
\usepackage{amsthm}
\usepackage{amsmath} 

\usepackage[serbian]{babel}

\newtheorem{thm}{Teorema}[section]
\theoremstyle{definition}
\newtheorem{dfn}{Definicija}[section]
\theoremstyle{remark}
\newtheorem{no}{Napomena}[section]
\theoremstyle{plain}
\newtheorem{lem}[thm]{Lema}

% text width and height
\textwidth 16cm
\textheight 23cm

% distance from the top
\voffset -1.5cm

% distance from the left
\hoffset 0cm
\oddsidemargin 0mm

% distance from the bottom
\footskip 1.5cm

\linespread{1}

\setlength\parindent{0pt}

\title{Primena fazi logike u obradi slika}
\author{Jelena Mrdak, mi15021\\ Tijana Jevti\' c}

\begin{document}
\maketitle
\tableofcontents

\section{FCM}
\label{sec:FCM}
Fuzzy C-means (FCM) je jedan od najpopularnijih algoritama za fazi klasterovanje. U ovom poglavlju \' cemo ga najpre detaljno opisati, a zatim \' cemo ga iskoristiti za binarizaciju slike.\\

Cilj ovog algoritma je da skup $X=\{x_{1}, x_{2}, ..., x_{n}\}$ particioni\v se na $k$ delova (klastera) po 
nekom kriterijumu. Preciznije, kriterijum je minimizacija slede\' ce funkcije:
\begin{equation*}
 F(X, \bar{w}, \bar{c}) = \sum_{i=1}^{n}\sum_{j=1}^{k}w_{ij}^{m}\left\|x_{i}-c_{j}\right\|^{2},
\end{equation*}
gde $w_{ij}\in [0, 1]$ predstavlja pripadnost ta\v cke $x_{i}$ $j$-tom klasteru i $\sum_{j=1}^{k}w_{ij}=1$, dok je $c_{j}$ centroid $j$-tog klastera. Parametar $m$ predstavlja faktor fazifikacije i on se zadaje unapred.
U nastavku \' cemo preciznije odrediti ove koeficijente. Sada \' cemo samo ukratko opisati korake algoritma.\\

FCM je veoma sli\v can algoritmu k-means i sastoji se iz slede\' cih koraka:
\begin{itemize}
  \item Izabrati broj klastera $k$.
  \item Svakoj ta\v cki $x_{i}$ dodeliti koeficijente $w_{ij} \in [0, 1], j=1,2,..., k$.
  \item Ponavljati sve dok ne do\dj e do konvergencije:
    \begin{itemize}
      \item Izra\v cunati centroide za svaki klaster.
      \item A\v zurirati koeficijente.
    \end{itemize}
  \item Ta\v cku $x_{i}$ dodeliti klasteru kom najvi\v se pripada, tj. $r$-tom klasteru, gde je $w_{ir}=\max\limits_{j} w_{ij}$.
\end{itemize}

\begin{thm}
  Funkcija $F(X, \bar{w}, \bar{c})$ dosti\v ze minimum za:
  \begin{equation}\label{c}
    c_{j} = \frac{\sum\limits_{i=1}^{n} w_{ij}^m \cdot x_{i}}{\sum\limits_{i=1}^{n} w_{ij}^{m}}
  \end{equation}
  i
  \begin{equation}\label{w}
    w_{ij} = \frac{1}{\sum\limits_{u=1}^{k} \biggl(\frac{\left\|x_{i}-c_{j}\right\|}{\left\|x_{i}-c_{u}\right\|}\biggr)^{\frac{2}{m-1}}} 
  \end{equation}
\end{thm}
\begin{proof}
Kako treba odrediti ekstremum funkcije uz ograni\v cenje, koristi\' cemo Lagran\v zove mno\v zioce. Odredi\' cemo ekstremum slede\' ce funkcije:
\begin{equation*}
  J(X, \bar{w}, \bar{c}, \bar{\lambda}) = \sum_{i=1}^{n}\sum_{j=1}^{k}w_{ij}^{m}\left\|x_{i}-c_{j}\right\|^{2} - \sum_{i=1}^{n}\lambda_{i}\biggl(\sum_{j=1}^{k}w_{ij}-1\biggr).
\end{equation*}

Potrebno je zadovoljiti slede\' ce uslove:
  \begin{equation}\label{der_c}
  \frac{\partial J}{\partial c_{j}} = 0, 1\leq j \leq k
\end{equation}
  \begin{equation}\label{der_w}
  \frac{\partial J}{\partial w_{ij}} = 0, 1\leq i \leq n, 1\leq j \leq k 
\end{equation}

  Re\v savanjem (\ref{der_c}) dobijamo (\ref{c}). Iz (\ref{der_w}) imamo
\begin{equation*}
  mw_{ij}^{m-1}\left\|x_{i}-c_{j}\right\|^{2} - \lambda_{i} = 0,
\end{equation*}
odnosno
\begin{equation}\label{wij}
  w_{ij} = \biggl(\frac{\lambda_{i}}{m\left\|x_{i}-c_{j}\right\|^{2}}\biggr)^{\frac{1}{m-1}}.
\end{equation}

Koriste\' ci po\v cetni uslov
\begin{align*}
  1&=\sum_{u=1}^{k}w_{iu}\\
   &=\sum_{u=1}^{k}\biggl(\frac{\lambda_{i}}{m\left\|x_{i}-c_{u}\right\|^{2}}\biggr)^{\frac{1}{m-1}}\\
   &=\sum_{u=1}^{k}\biggl(\frac{m\left\|x_{i}-c_{u}\right\|^{2}}{\lambda_{i}}\biggr)^{\frac{1}{1-m}}\\
   &=\sum_{u=1}^{k}\frac{(m\left\|x_{i}-c_{u}\right\|^{2})^{\frac{1}{1-m}}}{\lambda_{i}^{\frac{1}{1-m}}}\\
   &=\frac{1}{\lambda_{i}^{\frac{1}{1-m}}}\sum_{u=1}^{k}(m\left\|x_{i}-c_{u}\right\|^{2})^{\frac{1}{1-m}},
\end{align*}
zaklju\v cujemo da je
\begin{equation*}
  \lambda_{i} = \biggl(\sum_{u=1}^{k}(m\left\|x_{i}-c_{u}\right\|^{2})^{\frac{1}{1-m}}\biggr)^{1-m}.
\end{equation*}

Kona\v cno, zamenjuju\' ci poslednju jednakost u (\ref{wij}), dobijamo (\ref{w}). Ovime smo dokazali da se ekstremum dosti\v ze za date vrednosti. Dokaz da je u pitanju minimum \' cemo ovde presko\v citi.
\end{proof}

\end{document}
